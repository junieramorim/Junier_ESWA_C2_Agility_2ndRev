\begin{table}[ht!]
	\small
	\fontsize{10}{10}\selectfont
	\centering
	\caption{Qualitative (M1, M2, M3, M4 and M5), according to \cite{CC01}, and Quantitative (M6, M7 and M8) metrics used to evaluate the proposal}
	\label{table:metrics}
	
	\begin{tabularx}{\textwidth}{llX}
	\hline
		\textbf{ID}
		& \textbf{Metric}
		& \textbf{Description} \\ [1ex]
	\hline	
	
	M1 & Robustness & Ability of keeping the effectiveness level across different missions and circumstances. This effectiveness is the indicator to measure the C2 System's robustness.
	\\[1ex] \\
	
	M2 & Resilience & The capability of recovering after some issue or damage. The system looks for a stabilization after something that causes perturbation. It's correlated with responsiveness and flexibility.  
	\\[1ex] \\
	
	M3 & Responsiveness & The C2 system ability to act or react according to some context change in a timely manner. There is no unique optimal response and it depends on the resources and circumstances. However it needs to occur within reasonable time frame.
	\\[1ex] \\
	
	M4 & Flexibility & The capability to act in different ways to achieve the success. The identification of different paths of execution shows the flexibility level of the system. 
	\\[1ex] \\
	
	M5 & Adaptability & The capability of the system to change its organization or work process to become compatible with the new circumstance. It collaborates with the flexibility and responsiveness.
	\\[1ex] \\
	
	M6 & Effectiveness & Percentage of successful tasks completed by the executors.
	\\[1ex] \\
	
	M7 & Reconfigurations & Number of configurations performed by the members to adapt themselves to a new circumstance.
	\\[1ex] \\
	
	M8 & Adaptations & Number of C2 Maneuvers performed by the set of members in order to deal better with a new circumstance getting a new awareness level.
	\\[1ex]
	\hline
	\end{tabularx}
\end{table} 