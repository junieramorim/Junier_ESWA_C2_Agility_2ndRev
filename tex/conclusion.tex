C2 refers to mobilizing entities' efforts to the achievement of a goal and has been applied in a myriad of domains, across which dynamism abounds. A key requirement in these applications is C2 agility, which is the capability of the entities to deal with dynamic situations,  managing resources to achieve a timely and suitable C2 system response. Nevertheless, there is a lack of solutions on how to provide C2 agility to deal with dynamic contexts.

To demonstrate how C2 agility may be handled, we propose a computational model of a C2 System composed of entities collaborating towards completing a mission. The model defines how entities' reconfiguration and coordination may help handle context changes occurring in the mission, in the environment, or in the entities themselves. The C2 System model is formalized as a typed-parameterized extension of a channel system, which defines the roles and responsibilities handled by the entities constituting a C2 System, where each entity is modeled as a dynamic software product line. 


%This work extends the Channel System concept to propose a C2 Agility model based on behavior representation using Program Graphs. The entities have one or more different overlapping roles, and they can exchange information among them. This typed-parameterized model combines the reconfiguration and the coordination capabilities of the entities that play different roles, which are represented as program graphs. Besides, all artifacts related to our study are publicity available~\footnote{\gitRepository}.

%This typed-parameterized model uses channels to provide the necessary data and message exchange between roles. Such communication is essential to provide the coordination defined as a C2 principle. Furthermore, the Software Product Line (SPL) approach applied to model the members, implementing them as DSPL, guaranteed a self-configuration to optimize resources application to deal with context changes and to keep performing the mission. As SPL, the entities' onboard software is generated from features activated or not, reusing quality and previous structure. 


Simulation results suggest that the proposed model exhibits C2 agility. Indeed, the system spends more time in action, completes more tasks and more compatible ones, resulting in higher resilience thereby better coping with context changes and perturbations in dynamic scenarios.

Despite the shown evidence, this work presents some limitations that are related to the task allocation and the C2 Approach choice. The simulator uses a non-optimal algorithm based on swarm-gap strategy, with limited resilience, to perform a suitable allocation according to the context information available in the token. Moreover, the selection of the C2 Approach occurs sequentially according to its communications structure. Such a choice does not take into account a previous analysis of the context. In this case, the several attempts of the C2 Approach have a cost in terms of operating time and onboard resources, impacting the obtained results.

Furthermore, the C2 concepts are applied only in the computational environment, i.e., simulations. This application scenario may highlight restrictions to the extension of the proposed model to other domains and scenarios. Besides, the events that provide dynamism to the context considered in this work are suitable for UAV operation scenarios. To extend the proposed model to other areas, making it more compatible with real scenarios, it is necessary to include new events that may impact the model's behavior and may require significant adjustments.

Future work can explore the use of elaborated solutions applied to the function of maneuvering search and task allocation, \X{including the scenarios with restricted or compromised communications.} One possibility is the use of machine learning to process data collected from sensors in order to select the most suitable configuration and coordination structure. This Artificial Intelligence (AI) system can be improved with human activities controlling the start of any action, i.e., human-in-the-loop (HITL) strategy, or something more autonomous as human-on-the-loop~\citep{HITL01}. The proposed model can absorb any of these strategies to provide better results related to the mission execution in a dynamic context. Such a human collaboration is relevant because some C2 functions are related to behavior, e.g., leadership. In addition, other metrics, in particular those related to agility, e.g., robustness and innovation, can be inserted to make a deeper analysis regarding agility. \X{In addition, it is possible to perform a broader number of experiments with different scenarios, enlarging the domain scope with different number of entities, tasks, or changing events, in order to evaluate their effects and the system’s response to face a larger and more distinct number of context changes. Thus, we may insert new events into the CS through new actions that cause the expected effect on the system according to the context change that occurred.}

%Future work can explore the use of elaborated solutions applied to the function of maneuvering search and task allocation. One possibility is the use of machine learning to process data collected from sensors in order to select the most suitable configuration and coordination structure. This strategy can be plugged into the proposed model to provide better results related to the mission execution in a dynamic context. In addition, other metrics, in particular those related to agility, e.g., robustness and innovation, can be inserted to make a deeper agility analysis. 
