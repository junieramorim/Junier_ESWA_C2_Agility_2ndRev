\citet{FRANCE2014}, in the North Atlantic Treaty Organization (NATO) context, performed research to improve C2 agility in military operations. To collect evidence to support some identified hypotheses, they leveraged simulation-based experiments and retrospective case studies of real operations. Similar to our study, the simulations created a dynamic context to assess the entities' response. Differently, they did not address any cost notion in the simulated environment. Considering  time, entities' fuel, or other resources is essential because they influence the  obtained results in the mission accomplishment, in the worst case  making it unfeasible. Such analysis was left as an opportunity for future work and this is the focus of our study.

\citet{Swart2012} shows C2 as a set of processes that manipulate information and data through a network structure and does not describe the entities themselves involved with such processes.~\citet{Mason2001} proposed a software architecture that is capable of modeling a headquarters applying C2 processes in a military operation. Those studies, however, did not address the system response level in a dynamic context, which is addressed in our work.

\citet{Stanton2007} presented Process Model as a structure that summarizes the monitoring, decision making,  adaptation, and action executed by the C2 process to deal with context changes. Nonetheless, such a model does not show entities' behavior and relation during the mission under context change effects. Based on this and the concept of C2 Agility presented by~\citet{Alberts10}, our study proposes a model applied to the C2 process to deal with new constraints and conditions caused by dynamic contexts. 

\citet{c2-02} provided simulation-based evidence that adaptive networks can provide C2 Agility. 
Such work addresses only the network structure, not exploring the members' adaptation in case of circumstance changes. In contrast, our work proposes a model that addresses C2 Agility in two levels, one in terms of network structure and the second where there is an adaptation by member reconfiguration or task reallocation.

\citet{SAS04} shows that Self-Adaptive Systems (SAS) can adapt themselves to satisfy new requirements from the environment or the user. This characteristic has similar principles of C2 Agility where adaptation is a requirement to deal with context changes. However, this adaptation is only related to the systems' configuration. There is no coordination aspect involved, which would be related to C2 Maneuver Agility.

\citet{Rosenmuller2011} describe a Dynamic Software Product Line (DSPL) as an approach to model a SAS.
Moreover, \citet{BSN-DSPL} present techniques to ensure the reliability of such DSPL, which is a relevant quality aspect in critical domains.
These product-line design and verification methods can be leveraged to systematically handle the reconfiguration of team members in our setting.
However, we consider that the specifics of such techniques are implementation details, which are abstracted by functions \texttt{reconfig} and \texttt{find\_configuration} in the program graph for the TP role (Figure \ref{fig:ex_pg}).


According to~\citet{Rutten2017}, an automaton can be used to model SAS behavior. Besides, multiple automata can be connected forming a network and representing a set of components. \citet{MC01} show that such automata can be written as a Transition System (TS) with a finite number of states represented by Labeled Kripke Structures. As \citet{ltl02} use TS to model DSPLs, we may apply such structure to model C2 entities. However, modeling directly in this structure can become unfeasible due to a prohibitive number of states. To address this limitation, \citet{MC01} present the program graph abstraction.   We leverage this to model the behavior of entities that are part of a C2 System. We further leverage the parallel composition --channel system-- of these program graphs with a type-parameterized extension to achieve C2 Agility. 

%these program graphs are used to model roles, i.e., an abstraction of the entities' behavior that is part of a C2 structure. An agent can play one or more instances of a role. Based o this, \citet{agent0010} shows the use of roles as components of agents' implementation for modeling communication and coordination of an agent-based system~\citep{agent1}. These agents can accumulate different types of roles and responsibilities and execute them according to the circumstance~\citep{weyns2019activforms}.


Empirical studies with the use of simulation are widely explored and they are relevant to many different domains, e.g., military~\citep{CC03} and environmental monitoring~\citep{simulation001}. The simulation can create several circumstances to evaluate a solution or product, or to train professionals, reducing the need for resources to create real circumstances. In line with this philosophy and based on the comparison presented by~\citet{SIMUL01}, we have chosen Repast Simphony as the simulation environment and tool to perform tests and collect evidence related to our theory due to the technology usability and previous experience with the underlying Java technology. In general, the implementation of this work was based on strategies presented by~\citet{MAS07} and \citet{UAV01} where UAVs perform a reconnaissance mission. However, we extended the agent-based simulation to include the Maneuver Agility concept combined with team members' reconfiguration.

