Command and Control (C2) is a broad concept that encompasses the coordination of individuals and organizations towards achieving a goal.
However, dynamic and uncertain scenarios, such as military and disaster relief operations, present an inherent challenge to C2 activities.
In such situations, plans often need to be changed in the face of unforeseen problems, and even coordination processes may be subject to variation.
This dynamism increases the complexity of \color{black}resource management \color{black} and requires C2 Agility---i.e., the ability to respond to change in a timely and suitable fashion.
Nonetheless, there is a lack of solutions to provide C2 Agility to cope with dynamic contexts.
%
To address this problem, this work proposes a computational model of C2 Agility for a team of autonomous agents.
This model describes how to combine reconfiguration of individual team members and of coordination approaches to adapt to context changes.
%
The proposed approach leverages a typed-parameterized extension of a channel system to define the coordinating roles and responsibilities of team members.
Each member is modeled as a dynamic software product line, with the inherent ability to reconfigure itself.
%
\color{black}
To assess this model, a team of Unmanned Aerial Vehicles (UAV)
\color{black} performing a reconnaissance mission was simulated.
The simulation showed that the proposed model was suitable for dealing with dynamic contexts.
Particularly, metrics for the agile approach suggest improved system resilience in the face of induced perturbations, compared to non-agile C2.
%
The obtained results with the proposed software-based simulations showed that the proposed model is useful in providing C2 Agility to the studied scenarios, making the behavior of the entities specified in the model capable of dealing with context changes.


%It remains open for future work the possibility of enriching the dynamic scenarios with new events of context changes. In addition, other metrics application can also be considered to improve the system's agility level.


%Context: C2, C2 agility (Slides 2-12)

%\textbf{Context:} Created in the military domain, Command and Control (C2) is a definition which brings together a set of processes that seek to apply available resources to a mission accomplishment or a goal achievement. Nowadays, this concept has been applied in several domains, such as civil defense in disaster control, public security, and pandemics. In all these scenarios, changes in the context are inherent and they increase the complexity of resources managing. This dynamism requires a timely and suitable C2 system response response so-called C2 agility. Based on this, there is a lack of how to provide C2 agility to deal with dynamic contexts, especially in software-based simulations.

%Problem: There is no evidence on how to provide C2 Agility  (Slides 15-17)

%Relevance of the problem: addressing the problem is essential for effectiveness of C2 in realistic scenarios

%Solution: Computational model (Slides 19-26)

%Evaluation: Simulation (write in one sentence the Goal of slide 67) + The results show that


