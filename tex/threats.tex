%Threats to validity

Since the simulation presented in Section~\ref{sec:evaluation} was performed to assess the proposed model in Section~\ref{sec:proposal}, the following mitigation strategies were adopted to address the corresponding threats to validity: 

\begin{itemize}
   \item \textbf{Conclusion validity}: The execution scenarios involve random variables that provide different results with significant dispersion. To account for variance in the experiment environment, each factorial experiment combining scenarios and action methods was executed 500 times to reduce the standard deviation. Besides, in cases where the metrics for different action methods (A1 and A2) were equal to each other, a non-parametric test was applied to identify these minor differences among the results obtained to assess the system embedded agility. Such a test was chosen due to the non-normal distribution of the samples obtained from the experiments.
   
   
   \item \textbf{Internal validity}: The CS implementation performed by this work can obtain different results according to the allocation algorithm used and the strategy of C2 approach selection. To mitigate this threat, some well-known strategies were followed to allocate tasks inspired in swarm intelligence~\citep{MAS07, UAV01}. To perform the C2 Approach change, we followed the maturity scale proposed by~\citet{nato01}. This strategy is well known and recognized by the military domain in many real and simulated scenarios~\citep{FRANCE2014}. Furthermore, a simple implementation was performed to avoid technology dependence in results.
   
   
   \item \textbf{Construct validity}: To assure that the metrics chosen for the evaluation are suitable measures of the issue under investigation, they were derived from the C2 Agility attributes presented by~\citet{Alberts2006} and~\citet{nato01}. Additionally, the event changes that characterize a dynamic scenario are grounded on real military scenarios~\citep{UAV_Aplication, Power01}, and confirmed with domain experts.
   
  % Reviewer 2: Seems like another approach would be to come up with distributions of different kinds of events (perhaps provided by domain experts when actual data from real-world operations are not available) and randomly select the sequence based on those distributions. One wonders how such an approach would effect the results, or even how sensitive the results might be to the distributions. Maybe add some discussion of this if you think it would provide additional weight to these validity arguments.
  % The events’ order or even their kinds could affect the system’s response and it might become the analysis harder, generating outliers or not normalized results. To provide better conditions for analyzing the system response, we order the events under rules defined by domain experts to simulate a complexity increasing scenario.
  % To provide better conditions for analyzing the system response, we order the events under rules defined by domain experts to simulate a complexity increasing scenario.
   
   \item \textbf{External validity}: The events shown in Table~\ref{table:context_changes} are a small set of possible changes and perturbations under which a real system can be placed. Additionally, the sequence of changes used to create the scenarios and described in Table~\ref{tab:scenarios} can have a countless number of possibilities. To mitigate such a threat, this work considers events and increasingly complex scenarios that are well known in the military domain. \X{Based on this, the events’ order or even their kinds could affect the system’s response and it might become the analysis harder, generating outliers or not normalized results. A previous experiment with this set of events randomly chosen showed random results distribution and was difficult to analyze.} Besides, the role-based modeling used by this work explores the organization of responsibilities among entities and such strategy is applied by other domains cited in Section~\ref{sec:example}. Moreover, due to the fact that it involves elements related to human behavior, such as leadership~\citep{Alberts2006}, normally the strategies adopted to deal with this dynamism focus on historical data or hit-and-miss based on the context information available in order to provide a timely response. In general, it is possible to observe low maturity regarding to C2, given its complexity and multi-disciplinarity. In interviews and by collecting information from experts in the military domain (Brazilian Army), they confirmed the compatibility of the adopted treatments, i.e. A1 and A2, and realistic scenarios. Additionally, the results in ~\citet{UAV01} confirms the effects of dynamism using simulations and it uses a procedure similar to the A1 strategy. 
   % RELACIONADO À EXTRAPOLAÇÃO (APLICAÇÃO EM OUTROS CENÁRIOS)
   % FAZER O PARALELO DO DOMÍNIO MILITAR E A APLICAÇÃO EM OUTROS DOMINIOS CITADOS NA MOTIVAÇÃO DO TRABALHO.
   
\end{itemize}


